\documentclass[aspectratio=169,obeyspaces,spaces,hyphens,dvipsnames]{beamer}
\usepackage[utf8]{inputenc}
\usepackage{lmodern}% http://ctan.org/pkg/lm
\usepackage{minted}
\usepackage{hyperref}
\usepackage{xcolor}
\usepackage{pgfplots}
\usepackage{tikz}
\usepackage[normalem]{ulem}
\usepackage{textcomp}
\usepackage{media9}

\mode<presentation>
\usetheme{Bootlin}

\def\signed #1{{\leavevmode\unskip\nobreak\hfil\penalty50\hskip2em
  \hbox{}\nobreak\hfil(#1)
  \parfillskip=0pt \finalhyphendemerits=0 \endgraf}}

\newsavebox\mybox
\newenvironment{aquote}[1]
  {\savebox\mybox{#1}\begin{quotation}}
  {\signed{\usebox\mybox}\end{quotation}}

\title{MTD: What's new?}
\conference{Embedded Linux Conference Europe, October 2019}
\authors{Miquèl Raynal \hfill Richard Weinberger}
\email{miquel@bootlin.com \hfill richard@nod.at}
\institute{Bootlin \hfill Sigma-star GmbH}
\slidesurl{https://bootlin.com/pub/conferences/2019/elce/raynal-weinberger-mtd-what-s-new}

\begin{document}

\addtocontents{toc}{\protect\setcounter{tocdepth}{-1}}
\section{MTD: What's new?}
\addtocontents{toc}{\protect\setcounter{tocdepth}{2}}

\subsection{MTD}

\begin{frame}{Administration}
  \begin{itemize}
  \item Lot of changes in the maintainers team
    \begin{itemize}
    \item - Boris (MTD/NAND), David (MTD), Brian (MTD), Adrian
      (UBIFS), Artem (UBI/UBIFS), Marek (MTD), Cyrille (SPI-NOR)
    \item + Tudor (SPI-NOR), Vignesh (MTD/Hyperbus), Frieder (MTD),
      Miquèl (MTD/NAND)
    \item Richard (MTD/NAND/UBI/UBIFS) as a pillar
    \end{itemize}
  \item Migration to a kernel.org repository:\\
    \sout{\code{https://git.infradead.org/linux-mtd.git}}\\
    \code{https://git.kernel.org/pub/scm/linux/kernel/git/mtd/linux.git}\\
    \begin{itemize}
    \item mtd/fixes
    \item mtd/next (MTD + parallel NORs + Hyperbus)
    \item nand/next (all NAND flavors)
    \item spi-nor/next (SPI-NORs only)
    \end{itemize}
  \item UBI/UBIFS development is still on an infradead repository:\\
    \code{https://git.infradead.org/ubifs-2.6.git}\\
  \end{itemize}
\end{frame}

\begin{frame}{NAND: Fat cleanup}
  \begin{itemize}
  \item Creation of a raw/ subdirectory
  \item Move of Onenand into the nand/ subdirectory
  \item Make a distinction between the controller and the flash chip on
    the other side of the bus!
  \item Code reorganization (internals, ONFI, JEDEC)
  \item Deprecation of several hooks, creation of a legacy structure
  \item No more MTD objects in the NAND world
  \end{itemize}
\end{frame}

\begin{frame}{NAND: Generic NAND layer}
  \begin{itemize}
  \item Common to all flavors of NAND chips
  \item TODO: find examples
  \end{itemize}
\end{frame}

\begin{frame}{NAND: ->exec\_op()}
  \begin{itemize}
  \item Deprecating ->cmdfunc(), ->cmd\_ctrl(), ->select\_*() hooks
  \item TODO: take slides from previous confs
  \item Replacing them with ->exec\_op()
  \end{itemize}
\end{frame}

\begin{frame}{NAND: Supporting more and more hardware}
  \begin{itemize}
  \item Raw NANDs should have died by now
  \item New controllers (or IP flavors):
    \begin{itemize}
    \item MTK, Tegra, STM32 FMC2, Amlogic, Brcmnand, Macronix
    \end{itemize}
  \item Cleaned, updated, migrated to use the new hooks:
    \begin{itemize}
    \item Marvell, OMAP2, Sunxi, GPMI, Qualcomm, Denali, VF610, FSMC, Ams-Delta
    \end{itemize}
  \item Not dead enough from my point of view :)
  \item What does my crystal ball says?
  \end{itemize}
\end{frame}

\begin{frame}{NAND: SPI memories 1/2}
  \begin{itemize}
  \item Introduction of the SPI-mem layer
  \item TODO: more details
  \item SPI-NAND framework build on top of that
  \item TOO: take old slides
  \end{itemize}
\end{frame}

\begin{frame}{NAND: SPI memories 2/2}
  \begin{itemize}
  \item Staging SPI-NAND driver removed
  \item Plenty of new chips already supported, keeps increasing!
    \begin{itemize}
    \item Winbod, Gigadevices, Toshiba, Micron, Macronix, Paragon
    \end{itemize}
  \item Very active field, could replace raw memories in the near future
    due to the parallelization of data sampling (dual-spi, quad-spi,
    octo-spi) abstracted by the TODO layer.
  \end{itemize}
\end{frame}

\begin{frame}{NAND: SPI memories 2/2}
  \begin{itemize}
  \item Staging SPI-NAND driver removed
  \item Plenty of new chips already supported, keeps increasing!
    \begin{itemize}
    \item Winbod, Gigadevices, Toshiba, Micron, Macronix, Paragon
    \end{itemize}
  \item Very active field, could replace raw memories in the near future
    due to the parallelization of data sampling (dual-spi, quad-spi,
    octo-spi) abstracted by the TODO layer.
  \end{itemize}
\end{frame}

\begin{frame}{NAND: A journey supporting Micron's hardware}
  \begin{itemize}
  \item Aka: why we tend not to have full trust with hardware
    manufacturer?
  \item TODO
  \end{itemize}
\end{frame}

\begin{frame}{SPI-NOR}
\end{frame}

\begin{frame}{Hyperbus}
\end{frame}

\begin{frame}{UBI}
\end{frame}

\begin{frame}{UBIFS}
\end{frame}

\begin{frame}{What next?}
  \begin{itemize}
  \item External ECC engines <3
  \item MLC NAND support (at least SLC mode?)
  \item DMA-able buffers
  \item ONFI for SPI-NANDs? Maybe not
  \item In my dreams: an MTD I/O scheduler, parallelized accesses
    \begin{itemize}
    \item Running upstream Linux in SSDs?
    \end{itemize}
  \end{itemize}
\end{frame}

\begin{frame}
  \centering
  \Huge
  Questions? Suggestions? Comments?\\
  \vspace{0.5cm}
  \huge
  Miquèl Raynal \& Richard Weinberger\\
  \large
  \vspace{0.5cm}
  \hspace{2.6cm}\code{miquel@bootlin.com} and \code{richard@nod.at}
  \vspace{0.5cm}
  \newline Slides under CC-BY-SA 3.0\\
  \scriptsize
  \url{https://bootlin.com/pub/conferences/2019/elce/raynal-weinberger-what-s-new}
\end{frame}

\end{document}
